\documentclass{article}
\usepackage{amssymb,amsmath}
\title{minalpha projection step rationale}
\author{Chao-Yuan Wu, Kathryn Mazaitis}
\begin{document}
\maketitle

Given weight $w_r$ for the restart edge and $w_i$ for other outgoing edges $i$, we assert that the ratio of the reset weight to the total outgoing edge weight is $\alpha$:

$$\frac{f(w_r)}{f(w_r) + \sum\limits_i f(w_i)} = \alpha$$

In Chao-Yuan Wu's initial implementation, we made the assumption that the edges $w_i$ would contain a single (generally fact- or graph-) feature, such that $\sum\limits_i f(w_i)$ is simply the number of outgoing edges $i$ times the wrapped weight $f(w_i)$. This permits the solution:

\begin{align*}
\frac{f(w_r)}{f(w_r) + \sum\limits_i f(w_i)} & =   \alpha\\
\frac{f(w_r)}{f(w_r) + n f(w_i)} & =   \alpha\\
f(w_i) &= \frac{(1-\alpha)f(w_r)}{n \alpha}\\
w_i &= f^{-1}\left(\frac{(1-\alpha)f(w_r)}{n \alpha}\right)
\end{align*}

Now we will expand our implementation to accept nodes with multiple outgoing non-reset features. Let $n_k$ be the number of edges labeled with feature $\theta_k$, such that the total non-reset outgoing weight becomes $\sum\limits_k n_kf(\theta_k)$:

\begin{align*}
\frac{f(w_r)}{f(w_r) + \sum\limits_k n_kf(\theta_k)} &= \alpha\\
\sum\limits_k n_kf(\theta_k) &= \frac{(1-\alpha)f(w_r)}{\alpha}
\end{align*}

We will maintain the ratios between outgoing non-reset features between the old feature values $\theta'$ and the new feature values $\theta$ by making the assertion

$$\frac{n_kf(\theta_k)}{\sum\limits_j n_jf(\theta_j)} = \frac{n_kf(\theta_k')}{\sum\limits_j n_jf(\theta_j')}$$

In the code, the expression $(z-\mathrm{rw})$ represents the current (constraint-violating) non-reset outgoing weight, where $\mathrm{rw}=f(w_r)$. Then we can substitute and solve:

\begin{align*}
\frac{n_kf(\theta_k)}{\sum\limits_j n_jf(\theta_j)} &= \frac{n_kf(\theta_k')}{\sum\limits_j n_jf(\theta_j')}\\
\frac{n_kf(\theta_k)}{\left[\frac{(1-\alpha)f(w_r)}{\alpha}\right]} &= \frac{n_kf(\theta_k')}{z-\mathrm{rw}}\\
f(\theta_k) &= \frac{f(\theta_k')(1-\alpha)f(w_r)}{\alpha(z-\mathrm{rw})}\\
\theta_k &= f^{-1}\left(\frac{f(\theta_k')(1-\alpha)\mathrm{rw}}{\alpha(z-\mathrm{rw})}\right)
\end{align*}

\end{document}
