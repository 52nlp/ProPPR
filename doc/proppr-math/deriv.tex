\documentclass[12pt]{article}

\usepackage{graphicx}
\usepackage{latexsym,algorithm,algorithmic}
\usepackage{times}
\usepackage{amsmath,amssymb,mathtools}
\usepackage{mathptmx}      % use Times fonts if available on your TeX system

\setlength{\textwidth}{7in}
\setlength{\textheight}{9in}
\setlength{\oddsidemargin}{-0.5in}
\setlength{\evensidemargin}{-0.5in}
\setlength{\topmargin}{-0.5in}


\newcommand{\weightvec}{\textbf{w}}
\newcommand{\edge}[2]{{u\rightarrow{}v}}
\newcommand{\edgeuv}{{\edge{u}{v}}}

\newcommand{\vek}[1]{\textbf{#1}}
\newcommand{\ddw}{\frac{\partial}{\partial\vek{w}}}
\newcommand{\M}{\textrm{M}}
\newcommand{\dM}{\textrm{dM}}
\newcommand{\df}{\textbf{df}}
\newcommand{\dt}{\textbf{dt}}
\newcommand{\vphi}{\vec{\phi}}

\begin{document}

\section{Derivation: PPR and its derivative}

Notation: $\vek{s}$ is seeds, $\M$ is transition matrix, $\vek{w}$ are
parameters, and $\vek{p}^\infty$ is the PPR stationary distribution,
$\vek{p}^t$ is a point in the power-rule iteration for computating
PPR, and $\vek{d}^t$ is the partial derivative of $\vek{p}^t$ wrt the
parameters $\vek{w}$.

\begin{eqnarray}
\vek{p}^{t+1} & \equiv & \alpha \vek{s} + (1-\alpha) \M \vek{p}^t \\
\vek{d}^t & \equiv &  \ddw \vek{p}^t
\end{eqnarray}

Note $\vek{p}^t$ and $\vek{d}^t$ have different dimensions: while
$\vek{p}^t_u$ is a scalar score for $u$ under PPR, $\vek{d}^t_u$ is a
vector, giving the sensitivity of that score to each parameter in
$\vek{w}$.

$\M$ is defined as follows.  There is a weight vector $\vek{w}$, and
for an edge $\edgeuv$, there is a feature vector $\vphi_{uv}$, which is
used to define a basic score $s_{uv}$ for the edge, which is passed
thru a squashing function $f$, e.g., $f(x)\equiv e^x$, and then
normalized to form $\M$.

\begin{eqnarray}
 s_{uv}   & \equiv & \vphi_{uv} \cdot \vek{w}  \\
 t_u      & \equiv & \sum_{v'} f(s_{uv'}) \\
 \M_{u,v} & \equiv & \frac{ f(s_{uv}) }{ t_u }
\end{eqnarray}

We can define $\vek{d}^t$ recursively:


\begin{eqnarray}
\vek{d}^{t+1}  & =  &  \ddw \vek{p}^{t+1} \\
               & =  &  \ddw \left( \alpha \vek{s} + (1-\alpha) \M \vek{p}^t \right) \\
               & =  &  (1-\alpha) \ddw \M \vek{p}^t \\
               & =  &  (1-\alpha) \left( (\ddw \M) \vek{p}^t  + \M \ddw\vek{p}^t \right) \\
               & =  &  (1-\alpha) \left( (\ddw \M) \vek{p}^t  + \M \vek{d}^t \right)
\end{eqnarray}

Now let's look at $\ddw\M$, which I'll denote $\dM$ below.  Note that
each $\dM_{uv}$ is a vector, again giving the sensitivity of the
weight $\M_{uv}$ to each parameter in $\vek{w}$.

\begin{eqnarray}
\dM_{uv}   & = & \ddw \frac{ f(s_{uv}) }{ t_u } \\
           & = & \frac{1}{t_u ^2} \left( t_u \ddw f(s_{uv}) - f(s_{uv}) \ddw t_u \right)
\end{eqnarray}

To continue this, define $\df$ and $\dt$ as the vectors

\begin{eqnarray}
 \df_{uv} & \equiv & \ddw f(s_{uv}) = f'(s_{uv}) \vphi_{uv} \label{eqn:fprime} \\
 \dt_{u}  & \equiv & \ddw t_u = \sum_{v'} \df_{uv'}
\end{eqnarray}

Note that $\df_{uv}$ has no more non-zero components than
$\vphi_{uv}$, so it is sparse, and $\dt_{u}$ is also sparse, but
somewhat less so.

We can continue the derivation as

\begin{eqnarray}
\dM_{uv}   & = & \frac{1}{t_u ^2} \left( t_u \ddw f(s_{uv}) - f(s_{uv}) \ddw t_u \right) \\
           & = & \frac{1}{t_u ^2} \left( t_u \df_{uv} - f(s_{uv}) \dt_u \right)
\end{eqnarray}

This is a little different from the old algorithm: there is always an
(implicit) reset with probability $\alpha$, and that reset probability
can be increased by learning by weighting the features for the
(explicit) reset links that are present already in the graph.  No min
$\alpha$!

\section{Computation}

\begin{table}
\hrule
\begin{enumerate}
\item For each node/row $u$
  \begin{enumerate}
  \item  $t_u = 0$
  \item  $\dt_u = \vek{0}$, an all-zeros vector
  \item For each neighbor $v$ of $u$
    \begin{enumerate}    
    \item $s_{uv} = \vek{w} \cdot \vphi_{uv}$,  a scalar.

      In detail: For $i\in \vphi_{uv}$: increment $s_{uv}$ by $\vek{w}_i\vphi_i$

    \item $t_u  +\!\!= f(s_{uv})$, a scalar
    \item $\df_{uv} = f'(s_{uv}) \vphi_{uv}$, a vector, as sparse as $\vphi_{uv}$

      In detail: For $i\in \vphi_{uv}$: set $\df_{uv,i} = \vphi_f * c$, where $c=f'(s_{uv})$

    \item  $\dt_u +\!\!= \df_{uv}$, a vector, as sparse as $\sum_{v'} \vphi_{uv'}$

      In detail: For $i\in \df_{uv}$: increment $\dt_{u,i}$ by $\df_{uv,i}$

    \end{enumerate}
  Now $t_u = \sum_{v'} f(s_{uv'})$ and $\dt_{u} = \sum_{v'} \df_{uv'}$
  \item For each neighbor $v$ of $u$ create the vector

\[ \dM_{uv}  = \frac{1}{t_u ^2} \left( t_u \df_{uv} - f(s_{uv}) \dt_u \right)  
\]

Or in detail: For $i\in\dt_u$: $\dM_{uv,i} = \frac{1}{t_u ^2} \left( t_u \df_{uv,i} - f(s_{uv}) \dt_u,i \right)$.  

There aren't any dimensions $i$ that are present in $\df_{uv}$ but not
$\dt_u$, since $\dt_u$ is a summation.  Also create the scalar

\[ \M_{uv}  = \frac{f(s_{uv})}{t_u} 
\]

  You can now discard the intermediate values like $t_u$, $\dt_u$, $\df_{uv}$.

  \end{enumerate}
\end{enumerate}
\caption{Computing $\M$ and $\dM$} \label{alg:mat}
\hrule
\end{table}

Since $\dM$ and $M$ are reused many times, we should compute and store
$M$ and $\dM$ first.  This will expand the size of the graph somewhat:
in particular, we will need to store, not only the active edges $u,v$
and their features $\vphi_{uv}$, but also $\dM_{uv}$, which includes
weights for all features of vertexes $v'$ that are siblings of $v$
(i.e., there is an edge $u,v'$).  I'm not sure how bad this will be in
practice: perhaps we should estimate it for some of our test cases.
After this, operations should run in time linear in the size of the
new, less sparse graph encoded in $\dM$.  I believe that this scheme
also makes operations like the APR learning more modular.

Computing $M$ and $\dM$ is one pass over the graph, shown in Table
\ref{alg:mat}.


Then you can start with $\vek{p}^0 = \vek{d}^0 = \vek{0}$ and iterate

\begin{eqnarray}
\vek{p}^{t+1}  & =  & \alpha \vek{s} + (1-\alpha) \M \vek{p}^t \\
\vek{d}^{t+1}  & =  & (1-\alpha) \left( \dM \vek{p}^t  + \M \vek{d}^t \right)
\end{eqnarray}

In more detail, the iteration for the updates on $\vek{p}$ are shown
in Table~\ref{alg:updates}.

\begin{table}
\hrule

Updating $\vek{p}$:
\begin{enumerate}
\item $\vek{p}^{t+1} = \vek{0}$
\item For each node $u$ 
  \begin{enumerate}
  \item $\vek{p}^{t+1}_u +\!\!= \alpha \vek{s}_u$
  \item For each neighbor $v$ of $u$
    \begin{enumerate}
    \item $\vek{p}^{t+1}_u +\!\!= (1-\alpha) \M_{uv} \vek{p}^t_v$
    \end{enumerate}
  \end{enumerate}
\end{enumerate}

Updating $\vek{d}$:
\begin{enumerate}
\item $\vek{d}^{t+1} = \langle \vek{0}, \ldots, \vek{0} \rangle$ --- i.e., for each node $u$ there is an all-zeros
  vector of weights.
\item For each node $u$ 
  \begin{enumerate}
  \item For each neighbor $v$ of $u$
    \begin{enumerate}
      \item For each $i$ in $\dM_{uv}$

      \[ \vek{d}^{t+1}_{u,i} +\!\!= (1-\alpha) \dM_{uv,i} \vek{p}^t_v \]
      \item For each $i$ in $\vek{d}^t_{v}$

      \[ \vek{d}^{t+1}_{u,i} +\!\!= (1-\alpha) \M_{uv} \vek{d}^t_{v,i} \]
    \end{enumerate}
  \end{enumerate}
\end{enumerate}
\caption{Updates for $\vek{d}$ and $\vek{p}$} \label{alg:updates}
\hrule
\end{table}

\section{Loss functions and lazy regularization}

For SRW each example is a triple $(\vek{s},P,N)$ where $\vek{s}$ is
the seed distribution, $P=\{a^1,\ldots,a^I\}$ are the positive (a-ok?)
examples, and $N=\{b^1,\ldots,b^J\}$ are the negative (bad?) examples.
I use $\vek{p}$ for the PPR distribution starting at $\vek{s}$, and
write $\vek{p}[u]$ for $\vek{p}_u$ if I run out of space for
subscripts.

The loss function is
\begin{equation}
L(\vek{w}) \equiv - \Bigg (\sum_{k=1}^I \log \vek{p}[a^k] + \sum_{k=1}^J \log (1 - \vek{p}[b^k]) \Bigg) + \mu R(\vek{w})
\end{equation}
where $R(\vek{w})$ is the regularization, eg
$R(\vek{w})\equiv||\vek{w}||^2_2$. This means loss increases as the objective function decreases, where the objective function is proportional to the probability of either hitting a positive solution or not-hitting a negative solution.  Once we have $\vek{d}$ it's easy
to compute the gradient of this as

\begin{eqnarray}
\ddw L(\vek{w}) & \equiv & - \Bigg (\sum_{k=1}^I \frac{1}{\vek{p}[a^k]}\vek{d}[a^k] 
                                  - \sum_{k=1}^J \frac{1}{1 - \vek{p}[b^k]}\vek{d}[b^k] \Bigg) + \mu \ddw R(\vek{w})
\end{eqnarray}

We can split this into two parts: the empirical loss gradient, which is

\[
-\Bigg (\sum_{k=1}^I \frac{1}{\vek{p}[a^k]}\vek{d}[a^k] 
     - \sum_{k=1}^J \frac{1}{1 - \vek{p}[b^k]}\vek{d}[b^k] \Bigg)
\]
and the regularization gradient, 
\[
 \mu \ddw R(\vek{w})
\]
If an example doesn't contain all features then the empirical gradient
will be a sparse vector, but the regularization gradient will be
dense.  So the following code might be more efficient for SGD than
just computation of the full gradient and taking a step in that
direction.

\begin{enumerate}
\item Maintain a ``clock'' counter $m$ which is incremented when each
  example is processed. Also maintain a history $\vek{h}_i$ which
  says, for each feature $i$, the last time $t$ an example containing
  $i$ was processed.
\item When a new example $(\vek{s},P,N)$ arrives at time $t$
  \begin{enumerate}
    \item For each feature active in the example, initialize it, if
      necessary, and them perform the regularization-loss gradient
      update $t-\vek{h}_i$ times.
    \item Peform the empirical-loss update, using 
      the new weights.
  \end{enumerate}
\item When you finish learning at final time $T$, consider every
  feature $i$, and perform the regularization-loss gradient update
  $T-\vek{h}_i$ times.  Then write out the final parameters.
\end{enumerate}

\section{Inference: PPR and APR}

Inference using requires only $\M$, which at theorem-proving time, is
computed on-the-fly.  The power-iteration version of PPR, which in the
codebase is called the \texttt{PPRProver}, simply iterates this step
until convergence (or for a fixed number of iterations), starting with
$\vek{p}^0=\vek{0}$.

\begin{eqnarray}
\vek{p}^{t+1} & \equiv & \alpha \vek{s} + (1-\alpha) \M \vek{p}^t \\
\end{eqnarray}

Breaking this down, the one-step update is the following.

\begin{enumerate}
\item $\vek{p}^{t+1} = \vek{0}$
\item For each key $u$ with non-zero weight in $\vek{s}$:
  \begin{enumerate}
  \item $\vek{p}^{t+1}[u]$ += $\alpha \vek{s}[u]$
  \end{enumerate}
\item For each key $u$ with non-zero weight in $\vek{p}^t$:
  \begin{enumerate}
  \item For each node $v$ near $u$ in $\M$:
\(
       \vek{p}^{t+1}[v] \mbox{~+=~} (1-\alpha) \M[u,v] \vek{p}^t[u]
\)
  \end{enumerate}
\end{enumerate}



\section{Architectural Comments}

\emph{ [Katie's comments inline and signed with -k]}

\noindent
The routines that are suggested are:
\begin{itemize}
\item Loading: Compute $\M$ and $\dM$ from a graph and $\vek{w}$.  It
  seems necessary, altho a little non-modular, to initialize and
  create new features while the graph is read in. \emph{ [Do this here, or in SGD? -k]} We should be able
  to estimate performance pretty well after this step.

Inputs: graph and current values of parameters $\vek{w}$.

Outputs: $\M$ and $\dM$, and a list of all active features $i$.

Parameters: $f$, $f'$; method to init new features $i$ for $f$, clock
time and lazy-regularization function.
 
\item Extended inference: compute $\vek{p}$ and $\vek{d}$.
 
Inputs: $\vek{s}$, $\vek{w}$, $\M$ and $\dM$.

Outputs: $\vek{p}$ and $\vek{d}$.

\item SGD: Initialize new features $i$, use clock time and
  lazy-regularization function to update their weights, and update
  $\vek{w}$ using the empirical (non-gradient) loss.

Inputs: $\vek{p}$, $\vek{d}$, and $P$, $N$.

Outputs: Modified $\vek{w}$.

\end{itemize}

\begin{table}
\hrule
\begin{verbatim}
class Graph {

  // names of features

  String[] featName;

  // space for labels for feature weights on edges, each is a
  // feature index and a weight

  int[] label_featIndex;
  double[] label_featWeight;

  // space for edges, each is a destination node and a
  // list of labels, which point into the label space

  int[] edge_dst;
  int[] edge_labels_lo;
  int[] edge_labels_hi;

  // space for list of neighbors of nodes, each 
  // points into the edge space

  int [] node_near_lo;
  int [] node_near_hi;

  // list of nodes in graph
  int node_lo, node_hi;

  // space for weight derivatives, which are structurally 
  // just like labels.  these maybe should be vectors since 
  // we don't know the size of this space in advance...?

  int[] deriv_featIndex;
  double[] deriv_featDeriv;

  // list of derivative features for M[u][euv]
  int[][] dM_lo;
  int[][] dM_hi;
}
\end{verbatim}
\caption{Proposed data structure for graph} \label{alg:data}
\hrule
\end{table}

\begin{table}
\hrule
\begin{verbatim}
for (int u = g.node_lo; u < g.node_hi; u++) {
  // euv is edge from u to v
  for (euv = g.node_near_lo[u]; euv < g.node_near_hi[u]; euv++) {
    int v = g.edge_dst[euv];
    // luvk is k-th label on edge euv
    for (luvk = g.edge_labels_lo[euv]; luvk = g.edge_labels_hi[euv]; luvk++) {
      int i = g.label_featIndex[luvk];
      double w = g.label_featWeight[luvk];
      ...
    }
  }
}
for (int d = g.dM_lo[u][euv]; d < g.dM_hi[u][euv]; d++) {
  i = g.deriv_featIndex[d];
  double v = g.deriv_featDeriv[d];
  // v is partial/(partial feature i) if M[u][v]
  ...
}
\end{verbatim}
\caption{Accessing the proposed data structure for graph} \label{alg:access}
\hrule
\end{table}

A proposed data structure which would be very efficient is shown in
Table~\ref{alg:data} and \ref{alg:access}.  Some things that aren't
clear now are:
\begin{itemize}
\item Where does the feature-index $\leftrightarrow$ feature-name
  symbol table go?  Probably not in the graph as shown here.
\item Is $\M$ and $\dM$ part of the graph or a different structure?
\end{itemize}
Also a clarification: \texttt{M} and \texttt{dM\_lo,dM\_hi}, are not
dense 2-matrixes, of size quadratic in the number of nodes, $n$.
Instead \texttt{M} is a length-$n$ array of variable-size arrays, and
\texttt{M[u]}, for node index \texttt{u}, is an array of size $m_u$,
where $m_u$ is the number of neighbors (edges away from) \texttt{u}.
And \texttt{dM\_lo,dM\_hi} are parallel structures.

A final note: the squashing function $f$ and its derivative $f'$ are
only used in computing $\df$ in Equation~\ref{eqn:fprime}.  One future
extension to ProPPR might be to have facts with weights defined by
parameters: e.g., facts like \texttt{sim(a,b)} which when used would
lead to an edge with a computed weight $s_{uv}$ based on an internal
set of parameters $\vec{\lambda}$ (think of a learned similarity
subroutine).  A modular way of dealing with this might be worth
thinking through.  Conceptually, $\vec{\lambda}$ is a part of
$\vek{w}$: one could have an interface which given an edge $u,v$ with
weight $s_{uv}$ returns $f'(s_{uv})\ddw s_{uv}$ in terms of components
of $\vec{\lambda}$.  For training you would also need to pass in an
update of the $\vec{\lambda}$ features to the learning subroutine....

\end{document}

